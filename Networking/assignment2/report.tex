%\documentclass[10pt, a4paper, twocolumn, fleqn]{article}
\documentclass[10pt, a4paper, onecolumn, fleqn]{article}
\usepackage{amsmath}
\usepackage{geometry}
%\geometry{top=3.5cm,bottom=3cm}
%\geometry{left=2.5cm,right=2.8cm,top=3.5cm,bottom=3cm}
\geometry{left=2cm,right=2cm,top=2.5cm,bottom=2.5cm}

\begin{document}
\title{Homework $\sharp$2}
\author{Li Jiaying}
\maketitle

\section*{2}
Prove that in the p2p network Chord, if every node, say j,
maintains a table of short cuts with m entries,
in which the i-th entry points to the successor ($j+2^{(i-1)}) \mod 2^m$,
then the lookup complexity of the network is O($\log n$),
where n is the number of nodes in the network.


Prove:

First, n $<$ $2^m$, otherwise there must be at least one node
which can not be research from others. (e.g. node $2^m$)

As there is a $\mod 2^m$, we can look these nodes as a circle
beginned from anyone. Thus the lookup process become that we pick up
the beginner node(As each one can be the beginner), named S,
then find a specific node, named Q, in the circle.

We divide the whole circle into m parts, 
with the i-th parts contains the nodes with number [$2^{i-1}$, $2^i$),
and the first part just contains the beginner node S.

So by the first step, we can get to know which group Q belongs to.
And the group's length is at most $2^{m-1}$.
So the next step we do the same thing, the maximun length become $2^{m-2}$.

We can do this process iteratively until we find Q. And it will take less than or equal to m steps, which is also less than or equal to $\log n$.

So the lookup complexity of the network is O($\log n$).


\end{document}
